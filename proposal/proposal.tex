%%%%%%%%%%%%%%%%%%%%%%%%%%%%%%%%%%%%%%%%%%%%%%%%%%%%%%%%%%%%%%%%%%
%%%%%%%% ICML 2014 EXAMPLE LATEX SUBMISSION FILE %%%%%%%%%%%%%%%%%
%%%%%%%%%%%%%%%%%%%%%%%%%%%%%%%%%%%%%%%%%%%%%%%%%%%%%%%%%%%%%%%%%%

% Use the following line _only_ if you're still using LaTeX 2.09.
%\documentstyle[icml2014,epsf,natbib]{article}
% If you rely on Latex2e packages, like most moden people use this:
\documentclass{article}

% use Times
\usepackage{times}
% For figures
\usepackage{graphicx} % more modern
%\usepackage{epsfig} % less modern
\usepackage{subfigure} 

% For citations
\usepackage{natbib}

% For algorithms
\usepackage{algorithm}
\usepackage{algorithmic}

% As of 2011, we use the hyperref package to produce hyperlinks in the
% resulting PDF.  If this breaks your system, please commend out the
% following usepackage line and replace \usepackage{icml2014} with
% \usepackage[nohyperref]{icml2014} above.
\usepackage{hyperref}

% Packages hyperref and algorithmic misbehave sometimes.  We can fix
% this with the following command.
\newcommand{\theHalgorithm}{\arabic{algorithm}}

% Employ the following version of the ``usepackage'' statement for
% submitting the draft version of the paper for review.  This will set
% the note in the first column to ``Under review.  Do not distribute.''
\usepackage[accepted]{icml2014} 
% Employ this version of the ``usepackage'' statement after the paper has
% been accepted, when creating the final version.  This will set the
% note in the first column to ``Proceedings of the...''
%\usepackage[accepted]{icml2014}


% The \icmltitle you define below is probably too long as a header.
% Therefore, a short form for the running title is supplied here:
%\icmltitlerunning{Project Proposal for CS6780}

\begin{document} 

\twocolumn[
\icmltitle{Project Proposal for CS6780}

% It is OKAY to include author information, even for blind
% submissions: the style file will automatically remove it for you
% unless you've provided the [accepted] option to the icml2014
% package.
\icmlauthor{Alice Paul (ajp336), Calvin Wylie (cjw278), David Lingenbrink (dal299)}{}

% You may provide any keywords that you 
% find helpful for describing your paper; these are used to populate 
% the "keywords" metadata in the PDF but will not be shown in the document
\icmlkeywords{boring formatting information, machine learning, ICML}

\vskip 0.3in
]


\section{Motivation}
We will be competing in the 2015 Microsoft Malware Classification Challenge hosted by Kaggle Inc.  The goal of the competition is to correctly classify malware into families based on file content and characteristics.  Malware is becoming much more prevalent and smarter at avoiding detection by anti-malware software.  Recently, malware authors have started modifying their files to look like many different classes of files.  This makes the problem of detecting malware much more difficult.  In order to effectively handle malware, it is helpful to classify them into families that capture their behavior and motivations.  Our problem is to learn how to classify malware to 9 given families by using a large training test.  In addition, there is a \$16,000 prize for the winner.
\section{Statement of Problem}
Given a training set of malware files that have been correctly labeled, our goal is to generate probabilities $p_{ij}$ that a file $i$ is in family $j$. Suppose $y_{ij}$ is 1 if file $i$ is in class $j$ and 0 otherwise. Then, the loss of our predictions on the test data set with $N$ files and $M$ families is
\[ \mathrm{log loss} = - \frac{1}{N} \sum_{i=1}^N \sum_{j=1}^M y_{ij} \log (p_{ij}) . \]
While we have access to these test files, Microsoft has the correct labels and will return our log loss for a given set of probabilities. 
The competition ends on April 17th with the winner determined by the lowest log loss. 
\section{General Approach}
We first want to familiarize ourselves with the structure/format of the data, and extract what we guess are important characteristics of the binary malware files and disassembly output, thus generating the features that we would like our machine learning algorithms to use.  Next, we would design a train-validate-test work flow that uses the extracted features to predict the required probabilities. We will use the methods we learn in class in designing the learning stages of our algorithm, possibly combining and innovating several different methods. We will also be checking the position of our test score on the competition leaderboard.
\section{Resources}
Microsoft provides a large training set of data (500 GB uncompressed), as well as a test set, and we will need sufficient memory and compute resources to work with a large amount of data.

\section{Schedule}
\begin{description}
	\item[2/28] We should have a good understanding of the structure of the data  Create basic software to extract features (possibly trivial) from the data.
	\item[3/15] Implement basic machine learning methods with non-trivial features. 
	\item[3/31] Implement more innovative hybrid machine learning methods.  
	\item[4/15] Experiment with algorithm to optimize score.
	\item[4/17] Submit final machine learning algorithm to Microsoft.
	\item[5/1] Presentation.
	\item[5/11] Finish final report. 
\end{description}

\label{submission}

% In the unusual situation where you want a paper to appear in the
% references without citing it in the main text, use \nocite
\nocite{BIG}

\bibliography{proposalBib}
\bibliographystyle{icml2014}

\end{document} 


% This document was modified from the file originally made available by
% Pat Langley and Andrea Danyluk for ICML-2K. This version was
% created by Lise Getoor and Tobias Scheffer, it was slightly modified  
% from the 2010 version by Thorsten Joachims & Johannes Fuernkranz, 
% slightly modified from the 2009 version by Kiri Wagstaff and 
% Sam Roweis's 2008 version, which is slightly modified from 
% Prasad Tadepalli's 2007 version which is a lightly 
% changed version of the previous year's version by Andrew Moore, 
% which was in turn edited from those of Kristian Kersting and 
% Codrina Lauth. Alex Smola contributed to the algorithmic style files.  
